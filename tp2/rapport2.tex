\documentclass[a4paper]{article}

%% Language and font encodings
\usepackage[frenchb]{babel}
\usepackage[utf8x]{inputenc}
\usepackage[T1]{fontenc}
\usepackage{minted} %compiler avec la commande -shell-escape
\usepackage{graphicx}

%% Todo List
\usepackage{enumitem,amssymb}
\newlist{todolist}{itemize}{2}
\setlist[todolist]{label=$\square$}
\usepackage{pifont}
\newcommand{\cmark}{\ding{51}}%
\newcommand{\xmark}{\ding{55}}%
\newcommand{\done}{\rlap{$\square$}{\raisebox{2pt}{\large\hspace{1pt}\cmark}}%
\hspace{-2.5pt}}
\newcommand{\wontfix}{\rlap{$\square$}{\large\hspace{1pt}\xmark}}
\newcommand*{\escape}[1]{\texttt{\textbackslash#1}}
\newcommand*{\escapeI}[1]{\texttt{\expandafter\string\csname #1\endcsname}}
\newcommand*{\escapeII}[1]{\texttt{\char`\\#1}}

%% Sets page size and margins
\usepackage[a4paper,top=3cm,bottom=2cm,left=3cm,right=3cm,marginparwidth=1.75cm]{geometry}
\setlength{\parskip}{.5em}

\newcommand{\HRule}{\rule{\linewidth}{0.5mm}}

%-------------------------------------------------------------------------------
% TITLE PAGE
%-------------------------------------------------------------------------------

\title
{
	\LARGE{NACHOS : Entrées/Sorties}
	\HRule \\ [0.5cm]
	\LARGE \textbf{\uppercase{Devoir 1}}
	\HRule \\ [0.5cm]
}

\author{Guillaume CHARLET \\ Kenji FONTAINE}

\begin{document}

\null  % Empty line
\nointerlineskip  % No skip for prev line
\vfill
\let\snewpage \newpage
\let\newpage \relax
\maketitle
\let \newpage \snewpage
\vfill
\break % page break

%-------------------------------------------------------------------------------
% Table of Contents
%-------------------------------------------------------------------------------

\tableofcontents
\newpage

%-------------------------------------------------------------------------------
% Introduction
%-------------------------------------------------------------------------------

\section{Description du projet}
Ce travail a été réalisé dans un cadre universitaire,  par des étudiants en
Master 1 informatique à l'université de Bordeaux. \\
Un système d'exploitation est un ensemble de programmes permettant de diriger
l'utilisation des ressources d'un ordinateur. Il assure la liaison entre
l'utilisateur, les applications et le matériel.
Nachos est un processus permettant d'émuler un système d'exploitation et du
matériel. \\
L'objectif de ce devoir est de mettre en place sous Nachos un système
d'entrées-sorties minimales, permettant d'exécuter de petits programmes.
L'ensemble des modifications apportées sont entre balises \#ifdef CHANGED et
\#endif. \\

%-------------------------------------------------------------------------------
% Bilan
%-------------------------------------------------------------------------------

\section{Bilan}

\subsection{Partie 1}
L'objectif de cette première partie est de permettre à l'utilisateur de créer et
de manipuler des threads utilisateur en utilisant des appels systèmes qui
permettront de manipuler ces threads (créer, lancer, terminer). Le thread ainsi
crée partage le même espace mémoire que le thread père (AddrSpace).
L'implémentation de ces fonctions est assez primitive et sera améliorée dans la
partie 2. Cette première partie s'est relativement bien passée et ne nous a
pas posé beaucoup de problème.

\subsection{Partie 2}
Cette partie a pour but d'améliorer les fonctions implémentées dans la partie 1.
Nous devons permettre la création de plusieurs threads et faire en sorte qu'ils
s'éxecutent bien. La gestion de plusieurs threads en utilisant une BitMap nous a
posé des difficultés et n'est pas fonctionnelle.

\subsection{Partie 3}
Nous n'avons pas eu le temps de traiter cette partie.

\subsection{Partie 4}
Même chose que pour la partie 3.

%-------------------------------------------------------------------------------
% Points délicats
%-------------------------------------------------------------------------------

\newpage
\section{Points délicats}

Nos problèmes viennent principalement de la partie 2. La gestion de plusieurs
threads en utilisant la BitMap n'est pas fonctionnelle. En lançant la création
de plusieurs threads dans le thread principal, nous n'arrivons qu'à lancer le
thread principal ainsi que le premier thread de tous les threads que nous
voulons créer.

%-------------------------------------------------------------------------------
% Tests
%-------------------------------------------------------------------------------

\section{Tests}

Les fichiers de tests se trouvent dans le dossier /code/test. Ils sont exécutables
en utilisant la commande suivante :

\begin{minted}{bash}
$ ~/nachos/code/userprog ./nachos -x ../test/<fichier_test>
\end{minted}

\subsection{Partie 1 et 2}

Ces deux parties étant liées, il n'y a qu'un seul fichier de test. Le fichier
est disponible dans le dossier /code/test. Il est éxecutable en utilisant la
commande ci-dessus. Nous observons que le thread principal printant "main" ainsi
que seul le premier thread crée dans la boucle for sont lancés.


%-------------------------------------------------------------------------------
% END
%-------------------------------------------------------------------------------

\end{document}
